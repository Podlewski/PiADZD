\documentclass[a4paper,11pt]{article}
\usepackage[verbose,a4paper,tmargin=2cm,bmargin=2.3cm,lmargin=2.5cm,rmargin=2.5cm]{geometry}
\usepackage[utf8]{inputenc}
\usepackage{polski}
\usepackage{amsmath}
\usepackage{amsfonts}
\usepackage{amssymb}
\usepackage{lastpage}
\usepackage{indentfirst}
\usepackage{verbatim}
\usepackage{graphicx}
\usepackage{hyperref}
\hypersetup{
    colorlinks = true,
    linkcolor = black,
    urlcolor = cyan
}
\usepackage{fancyhdr}
\usepackage{listings}
\usepackage{hyperref} 
\usepackage{booktabs}
\usepackage[table,xcdraw]{xcolor}
\usepackage{multirow}
\usepackage{tikz}
\usepackage{float}
\frenchspacing
\pagestyle{fancyplain}
\fancyhf{}

\usepackage{setspace}

\setlength{\abovecaptionskip}{0pt}
\setlength{\belowcaptionskip}{3pt}

\renewcommand{\headrulewidth}{0pt}
\renewcommand{\footrulewidth}{0.5pt}
\newcommand{\degree}{\ensuremath{^{\circ}}} 
% \fancyfoot[L]{Przetwarzanie i analiza dużych zbiorów danych: }
% \fancyfoot[R]{\thepage\ / \pageref{LastPage}}
\fancyfoot[L]{Przetwarzanie i analiza dużych zbiorów danych: \\Aleksandra Kowalczyk, Zbigniew Nowacki, Karol Podlewski}
\fancyfoot[R]{\phantom{a} \\ \thepage\ / \pageref{LastPage}}

\begin{document}

\begin{titlepage}
\begin{center}

\begin{tabular}{rcl}
\begin{tabular}{|r|}
\hline \\
\large{\underline{234073~~~~~~~~~~~~~~~~~~~~~~} }\\
\small{\textit{Numer indeksu}}\\
\large{\underline{Aleksandra Kowalczyk~~} }\\
\small{\textit{Imię i nazwisko}}\\\\ \hline
\end{tabular} 
&
\begin{tabular}{|r|}
\hline \\
\large{\underline{234102~~~~~~~~~~~~~~~~~~~~~~} }\\
\small{\textit{Numer indeksu}}\\
\large{\underline{Zbigniew Nowacki~~~~~~~} }\\
\small{\textit{Imię i nazwisko}}\\\\ \hline
\end{tabular} 
&
\begin{tabular}{|r|}
\hline \\
\large{\underline{234106~~~~~~~~~~~~~~~~~~~~~~} }\\
\small{\textit{Numer indeksu}}\\
\large{\underline{Karol Podlewski~~~~~~~~~~} }\\
\small{\textit{Imię i nazwisko}}\\\\ \hline
\end{tabular} 

\end{tabular}
~\\~\\~\\ 
\end{center}
\begin{tabular}{ll}
\LARGE{\textbf{Kierunek}}& \LARGE{Informatyka Stosowana} \\
\LARGE{\textbf{Stopień}}& \LARGE{II} \\
\LARGE{\textbf{Specjalizacja}}& \LARGE{Data Science} \\
\LARGE{\textbf{Semestr}}& \LARGE{2} \\\\
\LARGE{\textbf{Data oddania}}& \LARGE{2 grudnia 2020} \\\\\\\\\\\\\\
\end{tabular}

\begin{center}
\textbf{\Huge{Przetwarzanie i analiza\\}}
\textbf{\Huge{dużych zbiorów danych\\~\\~\\~}}
\textbf{\Huge{Zadanie 4}}
\end{center}

\end{titlepage}

\setcounter{page}{2}

\setstretch{1.4}
\newpage
\section{Cel zadania}
    Celem zadania było stworzenie programu, który implementuje reguły asocjacyjne za pomocą algorytmu A-priori. Program powinien -- na podstawie ostatnio przeglądanych przez użytkowników przedmiotów -- identyfikować wszystkie pary i trójki przedmiotów, które wystąpiły wspólnie w ramach pojedynczej sesji co najmniej 100 razy, a następnie obliczyć ufność dla stworzonych reguł asocjacyjnych.
    
    Miary istotności:
    \begin{equation}
        \text{wsparcie: } supp(X) = \cfrac{|\{t \in T: X \subseteq t\}|}{|T|}
    \end{equation}
    \begin{equation}
        \text{ufność: } conf(X \Rightarrow Y) = \cfrac{supp(X \cup Y)}{supp(X)}
    \end{equation}
    Gdzie $X$ i $Y$ to zbiory elementów transakcji, $t$ oznacza transakcję, a $T$ jest zbiorem transakcji.

\section{Opis implementacji}
    Do zaimplementowania zadania wykorzystano język Python 3 oraz API PySpark pozwalające wykorzystać możliwości Apache Spark w pythonowym kodzie. Wszystkie funkcje zostały zaimplementowane w programie, nie korzystano z zewnętrznych bibliotek.

\section{Uzyskane wyniki}
    Pełne wyniki zostały przedstawione w plikach \texttt{result\_doubles.txt} oraz \texttt{result\_triples.txt}. Wyniki zostały posortowane malejąco po poziomie ufności oraz leksykograficznie rosnąco.
    
    5 reguł o największym współczynniku ufności zostało przedstawionych w tabelach \ref{tab:dubs} i \ref{tab:trips}.
    % Czas - 20 sekund

    \begin{table}[H]
        \centering
        \caption{Reguły asocjacyjne dla dwójek z najwyższym stopniem zaufania}
        \label{tab:dubs}
        \begin{tabular}{@{}llc@{}}
            \toprule
            \multicolumn{1}{c}{Poprzednik} & \multicolumn{1}{c}{Następnik} & Zaufanie          \\ \midrule
            DAI93865   & FRO40251  & 1,0               \\
            GRO85051   & FRO40251  & 0,999176276771005 \\
            GRO38636   & FRO40251  & 0,990654205607477 \\
            ELE12951   & FRO40251  & 0,990566037735849 \\
            DAI88079   & FRO40251  & 0,986725663716814 \\ \bottomrule
        \end{tabular}
    \end{table}

    \begin{table}[H]
        \centering
        \caption{Reguły asocjacyjne dla trójek z najwyższym stopniem zaufania}
        \label{tab:trips}
        \begin{tabular}{@{}lllc@{}}
            \toprule
            \multicolumn{2}{c}{Poprzednik} & \multicolumn{1}{c}{Następnik} & Zaufanie   \\ \midrule
            DAI23334       & ELE92920      & DAI62779  & 1,0        \\ 
            DAI31081       & GRO85051      & FRO40251  & 1,0        \\ 
            DAI55911       & GRO85051      & FRO40251  & 1,0        \\ 
            DAI62779       & DAI88079      & FRO40251  & 1,0        \\ 
            DAI75645       & GRO85051      & FRO40251  & 1,0        \\ \bottomrule
        \end{tabular}
    \end{table}

\section{Analiza wyników}
    \begin{itemize}
        \item Produkt o identyfikatorze \texttt{FRO40251} pojawia się we wszystkich regułach asocjacyjnych o~wysokiej ufności dla dwójek oraz w 4 na 5 reguł dla trójek.
        \item Trójki przedmiotów posiadają więcej reguł o wysokiej ufności -- dla dwójek jej wartość spada o wiele szybciej pomimo ogólnie większej (co widoczne jest w plikach) liczby reguł.
    \end{itemize}

\end{document}
