\documentclass[a4paper,11pt]{article}
\usepackage[verbose,a4paper,tmargin=2cm,bmargin=2.33cm,lmargin=2.5cm,rmargin=2.5cm]{geometry}
\usepackage[utf8]{inputenc}
\usepackage{polski}
\usepackage{amsmath}
\usepackage{amsfonts}
\usepackage{amssymb}
\usepackage{lastpage}
\usepackage{indentfirst}
\usepackage{verbatim}
\usepackage{graphicx}
\usepackage{hyperref}
\hypersetup{
    colorlinks = true,
    linkcolor = black,
    urlcolor = cyan
}
\usepackage{fancyhdr}
\usepackage{listings}
\usepackage{hyperref} 
\usepackage{xcolor}
\usepackage{tikz}
\usepackage{float}
\frenchspacing
\pagestyle{fancyplain}
\fancyhf{}

\usepackage{setspace}

\renewcommand{\headrulewidth}{0pt}
\renewcommand{\footrulewidth}{0.5pt}
\newcommand{\degree}{\ensuremath{^{\circ}}} 
% \fancyfoot[L]{Przetwarzanie i analiza dużych zbiorów danych: }
% \fancyfoot[R]{\thepage\ / \pageref{LastPage}}
\fancyfoot[L]{Przetwarzanie i analiza dużych zbiorów danych: \\Aleksandra Kowalczyk, Zbigniew Nowacki, Karol Podlewski}
\fancyfoot[R]{\phantom{a} \\ \thepage\ / \pageref{LastPage}}

\begin{document}

\begin{titlepage}
\begin{center}

\begin{tabular}{rcl}
\begin{tabular}{|r|}
\hline \\
\large{\underline{234073~~~~~~~~~~~~~~~~~~~~~~} }\\
\small{\textit{Numer indeksu}}\\
\large{\underline{Aleksandra Kowalczyk~~} }\\
\small{\textit{Imię i nazwisko}}\\\\ \hline
\end{tabular} 
&
\begin{tabular}{|r|}
\hline \\
\large{\underline{234102~~~~~~~~~~~~~~~~~~~~~~} }\\
\small{\textit{Numer indeksu}}\\
\large{\underline{Zbigniew Nowacki~~~~~~~} }\\
\small{\textit{Imię i nazwisko}}\\\\ \hline
\end{tabular} 
&
\begin{tabular}{|r|}
\hline \\
\large{\underline{234106~~~~~~~~~~~~~~~~~~~~~~} }\\
\small{\textit{Numer indeksu}}\\
\large{\underline{Karol Podlewski~~~~~~~~~~} }\\
\small{\textit{Imię i nazwisko}}\\\\ \hline
\end{tabular} 

\end{tabular}
~\\~\\~\\ 
\end{center}
\begin{tabular}{ll}
\LARGE{\textbf{Kierunek}}& \LARGE{Informatyka Stosowana} \\
\LARGE{\textbf{Stopień}}& \LARGE{II} \\
\LARGE{\textbf{Specjalizacja}}& \LARGE{Data Science} \\
\LARGE{\textbf{Semestr}}& \LARGE{2} \\\\
\LARGE{\textbf{Data oddania}}& \LARGE{4 listopada 2020} \\\\\\\\\\\\\\
\end{tabular}

\begin{center}
\textbf{\Huge{Przetwarzanie i analiza\\}}
\textbf{\Huge{dużych zbiorów danych\\~\\~\\~}}
\textbf{\Huge{Zadanie 2}}
\end{center}

\end{titlepage}

\setcounter{page}{2}

\setstretch{1.25}
\tableofcontents

\newpage
\section{Cel zadania}

Celem zadania była implementacja algorytmu "Osoby, które możesz znać", który jest wykorzystywany przede wszystkim przez media społecznościowe do sugerowania nowych, potencjalnych znajomych.

Każdy użytkownik posiadający znajomych dostaje rekomendacje do 10 nowych użytkowników, którzy nie są jego znajomymi, a mają najwięcej wspólnych znajomych. W przypadku użytkowników nie posiadających znajomych, program powinien zwrócić pustą listę.

\section{Opis implementacji}

Do implementacji zadania wykorzystano język Python wraz z API PySpark, które umożliwiło skorzystanie z możliwości języka Apache Spark w pythonowym kodzie.
Działanie programu można przedstawić w następujących krokach:
\begin{enumerate}
    \item Z podanego pliku tekstowego wczytano dane zawierające identyfikatory użytkowników oraz identyfikatory ich znajomych
    \item Dla każdego użytkownika stworzono listę rekomendacji dla jego znajomych łącząc wszystkich ze sobą w pary - następnie usunięto każde wystąpienie pary znajomych, którzy już są znajomymi, po czym obliczono jak często dana para występuje
    \item Dla każdego użytkownika przypisano parę w postaci nowego sugerowanego znajomego oraz liczby wspólnych znajomych, którą następnie posortowano po liczbie wspólnych znajomych i identyfikatorze użytkownika
    \item Do tej listy dodano użytkowników, dla których algorytm nie wyznaczył żadnych rekomendacji
\end{enumerate}

\subsection{Wykorzystanie Apache Sparka}

W celu implementacji algorytmu wykorzystano kolekcje RDD (Resilient Distributed Datasets), będące podstawową kolekcją obiektów Sparka, które są przystosowane do pracy z BigData. Obiekty RDD były poddawane przeróżnym transformacjom takim jak filtracja czy mapowanie. Niezbędne okazało się także korzystanie z redukcji czy sortowania obiektów po kluczu - niezwykle przydatna okazała się redukcja po kluczu \verb|reduceByKey()|, która znacznie ułatwiła proces zliczania par o identycznych kluczach. 

\newpage

\section{Uzyskane wyniki}

W tabeli poniżej przedstawiono rekomendacje uzyskane dla użytkowników o identyfikatorach: 924, 8941, 8942, 9019, 9020, 9021, 9022, 9990, 9992 oraz 9993.

\begin{table}[H]
\centering
\caption{Użytkownicy wskazani w poleceniu oraz proponowani im znajomi}
\begin{tabular}{|c|l|}
\hline
Użytkownik & \multicolumn{1}{c|}{Rekomendacje} \\ \hline
924 & 439, 2409, 6995, 11860, 15416, 43748, 45881 \\ \hline
8941 & 8943, 8944, 8940 \\ \hline
8942 & 8939, 8940, 8943, 8944 \\ \hline
9019 & 9022, 317, 9023 \\ \hline
9020 & 9021, 9016, 9017, 9022, 317, 9023 \\ \hline
9021 & 9020, 9016, 9017, 9022, 317, 9023 \\ \hline
9022 & 9019, 9020, 9021, 317, 9016, 9017, 9023 \\ \hline
9990 & 13134, 13478, 13877, 34299, 34485, 34642, 37941 \\ \hline
9992 & 9987, 9989, 35667, 9991 \\ \hline
9993 & 9991, 13134, 13478, 13877, 34299, 34485, 34642, 37941 \\ \hline
\end{tabular}
\end{table}

\end{document}
